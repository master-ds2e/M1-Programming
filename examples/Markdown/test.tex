% Options for packages loaded elsewhere
\PassOptionsToPackage{unicode}{hyperref}
\PassOptionsToPackage{hyphens}{url}
%
\documentclass[
]{article}
\usepackage{lmodern}
\usepackage{amssymb,amsmath}
\usepackage{ifxetex,ifluatex}
\ifnum 0\ifxetex 1\fi\ifluatex 1\fi=0 % if pdftex
  \usepackage[T1]{fontenc}
  \usepackage[utf8]{inputenc}
  \usepackage{textcomp} % provide euro and other symbols
\else % if luatex or xetex
  \usepackage{unicode-math}
  \defaultfontfeatures{Scale=MatchLowercase}
  \defaultfontfeatures[\rmfamily]{Ligatures=TeX,Scale=1}
\fi
% Use upquote if available, for straight quotes in verbatim environments
\IfFileExists{upquote.sty}{\usepackage{upquote}}{}
\IfFileExists{microtype.sty}{% use microtype if available
  \usepackage[]{microtype}
  \UseMicrotypeSet[protrusion]{basicmath} % disable protrusion for tt fonts
}{}
\makeatletter
\@ifundefined{KOMAClassName}{% if non-KOMA class
  \IfFileExists{parskip.sty}{%
    \usepackage{parskip}
  }{% else
    \setlength{\parindent}{0pt}
    \setlength{\parskip}{6pt plus 2pt minus 1pt}}
}{% if KOMA class
  \KOMAoptions{parskip=half}}
\makeatother
\usepackage{xcolor}
\IfFileExists{xurl.sty}{\usepackage{xurl}}{} % add URL line breaks if available
\IfFileExists{bookmark.sty}{\usepackage{bookmark}}{\usepackage{hyperref}}
\hypersetup{
  pdftitle={Scheduling exercise solution},
  pdfauthor={Pierre Pelletier},
  hidelinks,
  pdfcreator={LaTeX via pandoc}}
\urlstyle{same} % disable monospaced font for URLs
\usepackage[margin=1in]{geometry}
\usepackage{color}
\usepackage{fancyvrb}
\newcommand{\VerbBar}{|}
\newcommand{\VERB}{\Verb[commandchars=\\\{\}]}
\DefineVerbatimEnvironment{Highlighting}{Verbatim}{commandchars=\\\{\}}
% Add ',fontsize=\small' for more characters per line
\usepackage{framed}
\definecolor{shadecolor}{RGB}{248,248,248}
\newenvironment{Shaded}{\begin{snugshade}}{\end{snugshade}}
\newcommand{\AlertTok}[1]{\textcolor[rgb]{0.94,0.16,0.16}{#1}}
\newcommand{\AnnotationTok}[1]{\textcolor[rgb]{0.56,0.35,0.01}{\textbf{\textit{#1}}}}
\newcommand{\AttributeTok}[1]{\textcolor[rgb]{0.77,0.63,0.00}{#1}}
\newcommand{\BaseNTok}[1]{\textcolor[rgb]{0.00,0.00,0.81}{#1}}
\newcommand{\BuiltInTok}[1]{#1}
\newcommand{\CharTok}[1]{\textcolor[rgb]{0.31,0.60,0.02}{#1}}
\newcommand{\CommentTok}[1]{\textcolor[rgb]{0.56,0.35,0.01}{\textit{#1}}}
\newcommand{\CommentVarTok}[1]{\textcolor[rgb]{0.56,0.35,0.01}{\textbf{\textit{#1}}}}
\newcommand{\ConstantTok}[1]{\textcolor[rgb]{0.00,0.00,0.00}{#1}}
\newcommand{\ControlFlowTok}[1]{\textcolor[rgb]{0.13,0.29,0.53}{\textbf{#1}}}
\newcommand{\DataTypeTok}[1]{\textcolor[rgb]{0.13,0.29,0.53}{#1}}
\newcommand{\DecValTok}[1]{\textcolor[rgb]{0.00,0.00,0.81}{#1}}
\newcommand{\DocumentationTok}[1]{\textcolor[rgb]{0.56,0.35,0.01}{\textbf{\textit{#1}}}}
\newcommand{\ErrorTok}[1]{\textcolor[rgb]{0.64,0.00,0.00}{\textbf{#1}}}
\newcommand{\ExtensionTok}[1]{#1}
\newcommand{\FloatTok}[1]{\textcolor[rgb]{0.00,0.00,0.81}{#1}}
\newcommand{\FunctionTok}[1]{\textcolor[rgb]{0.00,0.00,0.00}{#1}}
\newcommand{\ImportTok}[1]{#1}
\newcommand{\InformationTok}[1]{\textcolor[rgb]{0.56,0.35,0.01}{\textbf{\textit{#1}}}}
\newcommand{\KeywordTok}[1]{\textcolor[rgb]{0.13,0.29,0.53}{\textbf{#1}}}
\newcommand{\NormalTok}[1]{#1}
\newcommand{\OperatorTok}[1]{\textcolor[rgb]{0.81,0.36,0.00}{\textbf{#1}}}
\newcommand{\OtherTok}[1]{\textcolor[rgb]{0.56,0.35,0.01}{#1}}
\newcommand{\PreprocessorTok}[1]{\textcolor[rgb]{0.56,0.35,0.01}{\textit{#1}}}
\newcommand{\RegionMarkerTok}[1]{#1}
\newcommand{\SpecialCharTok}[1]{\textcolor[rgb]{0.00,0.00,0.00}{#1}}
\newcommand{\SpecialStringTok}[1]{\textcolor[rgb]{0.31,0.60,0.02}{#1}}
\newcommand{\StringTok}[1]{\textcolor[rgb]{0.31,0.60,0.02}{#1}}
\newcommand{\VariableTok}[1]{\textcolor[rgb]{0.00,0.00,0.00}{#1}}
\newcommand{\VerbatimStringTok}[1]{\textcolor[rgb]{0.31,0.60,0.02}{#1}}
\newcommand{\WarningTok}[1]{\textcolor[rgb]{0.56,0.35,0.01}{\textbf{\textit{#1}}}}
\usepackage{graphicx}
\makeatletter
\def\maxwidth{\ifdim\Gin@nat@width>\linewidth\linewidth\else\Gin@nat@width\fi}
\def\maxheight{\ifdim\Gin@nat@height>\textheight\textheight\else\Gin@nat@height\fi}
\makeatother
% Scale images if necessary, so that they will not overflow the page
% margins by default, and it is still possible to overwrite the defaults
% using explicit options in \includegraphics[width, height, ...]{}
\setkeys{Gin}{width=\maxwidth,height=\maxheight,keepaspectratio}
% Set default figure placement to htbp
\makeatletter
\def\fps@figure{htbp}
\makeatother
\setlength{\emergencystretch}{3em} % prevent overfull lines
\providecommand{\tightlist}{%
  \setlength{\itemsep}{0pt}\setlength{\parskip}{0pt}}
\setcounter{secnumdepth}{-\maxdimen} % remove section numbering
\usepackage{booktabs}
\usepackage{longtable}
\usepackage{array}
\usepackage{multirow}
\usepackage{wrapfig}
\usepackage{float}
\usepackage{colortbl}
\usepackage{pdflscape}
\usepackage{tabu}
\usepackage{threeparttable}
\usepackage{threeparttablex}
\usepackage[normalem]{ulem}
\usepackage{makecell}
\usepackage{xcolor}
\ifluatex
  \usepackage{selnolig}  % disable illegal ligatures
\fi

\title{Scheduling exercise solution}
\author{Pierre Pelletier}
\date{2/7/2021}

\begin{document}
\maketitle

{
\setcounter{tocdepth}{2}
\tableofcontents
}
\newpage

\hypertarget{exercice}{%
\section{Exercice}\label{exercice}}

There are XX groups in the class, the orals must be spread over 6 hours,
in two 3-hour sessions. For each group the duration of the oral is about
\textasciitilde20 min + \textasciitilde10 min for Q\&A. We will want to
take two 15-minute breaks in the after 1h30 of courses. Write a program
that randomly assigns a schedule to each group, write out this list of
names and schedules in a table. Have a look on this :
\url{https://cran.r-project.org/web/packages/kableExtra/vignettes/awesome_table_in_html.html}

\hypertarget{examples}{%
\section{Examples}\label{examples}}

\hypertarget{using-python}{%
\subsection{Using Python}\label{using-python}}

\begin{Shaded}
\begin{Highlighting}[]

\ImportTok{import}\NormalTok{ pandas }\ImportTok{as}\NormalTok{ pd}

\NormalTok{test\_py }\OperatorTok{=}\NormalTok{ pd.DataFrame(}
\NormalTok{    \{}
     \StringTok{\textquotesingle{}schedule\textquotesingle{}}\NormalTok{:[}\StringTok{\textquotesingle{}h1\textquotesingle{}}\NormalTok{,}\StringTok{\textquotesingle{}h2\textquotesingle{}}\NormalTok{,}\StringTok{\textquotesingle{}h3\textquotesingle{}}\NormalTok{],}
     \StringTok{\textquotesingle{}name\textquotesingle{}}\NormalTok{:[}\StringTok{\textquotesingle{}A\textquotesingle{}}\NormalTok{,}\StringTok{\textquotesingle{}B\textquotesingle{}}\NormalTok{,}\StringTok{\textquotesingle{}C\textquotesingle{}}\NormalTok{]}
\NormalTok{     \}}
\NormalTok{    )}
\end{Highlighting}
\end{Shaded}

\hypertarget{using-r}{%
\subsection{Using R}\label{using-r}}

\begin{Shaded}
\begin{Highlighting}[]
\KeywordTok{library}\NormalTok{(dplyr)}
\end{Highlighting}
\end{Shaded}

\begin{verbatim}
## 
## Attaching package: 'dplyr'
\end{verbatim}

\begin{verbatim}
## The following objects are masked from 'package:stats':
## 
##     filter, lag
\end{verbatim}

\begin{verbatim}
## The following objects are masked from 'package:base':
## 
##     intersect, setdiff, setequal, union
\end{verbatim}

\begin{Shaded}
\begin{Highlighting}[]
\NormalTok{test\_R =}\StringTok{ }\KeywordTok{data.frame}\NormalTok{(}
  \StringTok{\textquotesingle{}schedule\textquotesingle{}}\NormalTok{=}\KeywordTok{c}\NormalTok{(}\StringTok{\textquotesingle{}h1\textquotesingle{}}\NormalTok{,}\StringTok{\textquotesingle{}h2\textquotesingle{}}\NormalTok{,}\StringTok{\textquotesingle{}h3\textquotesingle{}}\NormalTok{),}
  \StringTok{\textquotesingle{}name\textquotesingle{}}\NormalTok{=}\KeywordTok{c}\NormalTok{(}\StringTok{\textquotesingle{}A\textquotesingle{}}\NormalTok{,}\StringTok{\textquotesingle{}B\textquotesingle{}}\NormalTok{,}\StringTok{\textquotesingle{}C\textquotesingle{}}\NormalTok{)}
\NormalTok{  )}
\end{Highlighting}
\end{Shaded}

\hypertarget{get-python-object-in-r}{%
\section{Get Python object in R}\label{get-python-object-in-r}}

\begin{Shaded}
\begin{Highlighting}[]
\KeywordTok{library}\NormalTok{(reticulate)}

\NormalTok{py}\OperatorTok{$}\NormalTok{test\_py}
\end{Highlighting}
\end{Shaded}

\begin{verbatim}
##   schedule name
## 1       h1    A
## 2       h2    B
## 3       h3    C
\end{verbatim}

\hypertarget{create-latex-tables}{%
\section{Create latex tables}\label{create-latex-tables}}

\begin{Shaded}
\begin{Highlighting}[]
\KeywordTok{library}\NormalTok{(kableExtra)}
\end{Highlighting}
\end{Shaded}

\begin{verbatim}
## 
## Attaching package: 'kableExtra'
\end{verbatim}

\begin{verbatim}
## The following object is masked from 'package:dplyr':
## 
##     group_rows
\end{verbatim}

\begin{Shaded}
\begin{Highlighting}[]
\NormalTok{py}\OperatorTok{$}\NormalTok{test\_py }\OperatorTok{\%\textgreater{}\%}\StringTok{ }
\StringTok{  }\KeywordTok{kbl}\NormalTok{(}\DataTypeTok{booktabs =}\NormalTok{ T,}
      \DataTypeTok{caption =} \StringTok{"Schedules using pandas\textquotesingle{} DF"}\NormalTok{) }\OperatorTok{\%\textgreater{}\%}
\StringTok{    }\KeywordTok{kable\_classic}\NormalTok{(}\DataTypeTok{full\_width =}\NormalTok{ F, }\DataTypeTok{html\_font =} \StringTok{"Cambria"}\NormalTok{)}
\end{Highlighting}
\end{Shaded}

\begin{table}

\caption{\label{tab:unnamed-chunk-4}Schedules using pandas' DF}
\centering
\begin{tabular}[t]{ll}
\toprule
schedule & name\\
\midrule
h1 & A\\
h2 & B\\
h3 & C\\
\bottomrule
\end{tabular}
\end{table}

\begin{Shaded}
\begin{Highlighting}[]
\NormalTok{test\_R }\OperatorTok{\%\textgreater{}\%}\StringTok{ }
\StringTok{  }\KeywordTok{kbl}\NormalTok{(}\DataTypeTok{booktabs =}\NormalTok{ T,}
      \DataTypeTok{caption =} \StringTok{"Schedules R\textquotesingle{}s DF"}\NormalTok{) }\OperatorTok{\%\textgreater{}\%}
\StringTok{    }\KeywordTok{kable\_classic}\NormalTok{(}\DataTypeTok{full\_width =}\NormalTok{ F, }\DataTypeTok{html\_font =} \StringTok{"Cambria"}\NormalTok{)}
\end{Highlighting}
\end{Shaded}

\begin{table}

\caption{\label{tab:unnamed-chunk-4}Schedules R's DF}
\centering
\begin{tabular}[t]{ll}
\toprule
schedule & name\\
\midrule
h1 & A\\
h2 & B\\
h3 & C\\
\bottomrule
\end{tabular}
\end{table}

\begin{itemize}
\tightlist
\item
  then we put \texttt{echo\ =\ FALSE,} in
  \texttt{\textasciigrave{}\textasciigrave{}\textasciigrave{}\{r,\ echo\ =\ FALSE\}\textasciigrave{}\textasciigrave{}\textasciigrave{}}
  and we don't see the input anymore
\end{itemize}

\begin{table}

\caption{\label{tab:unnamed-chunk-5}Schedules using pandas' DF}
\centering
\begin{tabular}[t]{ll}
\toprule
schedule & name\\
\midrule
h1 & A\\
h2 & B\\
h3 & C\\
\bottomrule
\end{tabular}
\end{table}

\begin{table}

\caption{\label{tab:unnamed-chunk-5}Schedules R's DF}
\centering
\begin{tabular}[t]{ll}
\toprule
schedule & name\\
\midrule
h1 & A\\
h2 & B\\
h3 & C\\
\bottomrule
\end{tabular}
\end{table}

\end{document}
