% Options for packages loaded elsewhere
\PassOptionsToPackage{unicode}{hyperref}
\PassOptionsToPackage{hyphens}{url}
%
\documentclass[
]{article}
\usepackage{lmodern}
\usepackage{amssymb,amsmath}
\usepackage{ifxetex,ifluatex}
\ifnum 0\ifxetex 1\fi\ifluatex 1\fi=0 % if pdftex
  \usepackage[T1]{fontenc}
  \usepackage[utf8]{inputenc}
  \usepackage{textcomp} % provide euro and other symbols
\else % if luatex or xetex
  \usepackage{unicode-math}
  \defaultfontfeatures{Scale=MatchLowercase}
  \defaultfontfeatures[\rmfamily]{Ligatures=TeX,Scale=1}
\fi
% Use upquote if available, for straight quotes in verbatim environments
\IfFileExists{upquote.sty}{\usepackage{upquote}}{}
\IfFileExists{microtype.sty}{% use microtype if available
  \usepackage[]{microtype}
  \UseMicrotypeSet[protrusion]{basicmath} % disable protrusion for tt fonts
}{}
\makeatletter
\@ifundefined{KOMAClassName}{% if non-KOMA class
  \IfFileExists{parskip.sty}{%
    \usepackage{parskip}
  }{% else
    \setlength{\parindent}{0pt}
    \setlength{\parskip}{6pt plus 2pt minus 1pt}}
}{% if KOMA class
  \KOMAoptions{parskip=half}}
\makeatother
\usepackage{xcolor}
\IfFileExists{xurl.sty}{\usepackage{xurl}}{} % add URL line breaks if available
\IfFileExists{bookmark.sty}{\usepackage{bookmark}}{\usepackage{hyperref}}
\hypersetup{
  pdftitle={Techniques de Programmation (6 ECTS)},
  pdfauthor={Pierre Pelletier},
  hidelinks,
  pdfcreator={LaTeX via pandoc}}
\urlstyle{same} % disable monospaced font for URLs
\usepackage[margin=1in]{geometry}
\usepackage{graphicx}
\makeatletter
\def\maxwidth{\ifdim\Gin@nat@width>\linewidth\linewidth\else\Gin@nat@width\fi}
\def\maxheight{\ifdim\Gin@nat@height>\textheight\textheight\else\Gin@nat@height\fi}
\makeatother
% Scale images if necessary, so that they will not overflow the page
% margins by default, and it is still possible to overwrite the defaults
% using explicit options in \includegraphics[width, height, ...]{}
\setkeys{Gin}{width=\maxwidth,height=\maxheight,keepaspectratio}
% Set default figure placement to htbp
\makeatletter
\def\fps@figure{htbp}
\makeatother
\setlength{\emergencystretch}{3em} % prevent overfull lines
\providecommand{\tightlist}{%
  \setlength{\itemsep}{0pt}\setlength{\parskip}{0pt}}
\setcounter{secnumdepth}{-\maxdimen} % remove section numbering
\ifluatex
  \usepackage{selnolig}  % disable illegal ligatures
\fi

\title{Techniques de Programmation (6 ECTS)}
\author{Pierre Pelletier}
\date{}

\begin{document}
\maketitle
\begin{abstract}
Ce pdf récapitule les modalités d'évaluation pour le cours de Techniques
de Programmation (M1 APE). Veuillez lire attentivement chacun de ces
points pour comprendre ce qui est attendu pour votre projet, oral et
écrit. Si certains points restent selon vous encore flous, n'hésitez pas
à le faire remonter.
\end{abstract}

{
\setcounter{tocdepth}{2}
\tableofcontents
}
\newpage

\hypertarget{examen-uxe9crit}{%
\section{Examen écrit}\label{examen-uxe9crit}}

\begin{itemize}
\item
  \textbf{16 Mars 9h-11h | Coeff : 1/3 (2ECTS)}
\item
  L'examen se déroulera sur ordinateur et en présentiel.
\item
  Prévoyez votre ordinateur, mais j'ai demandé à avoir une salle info
  (je ne pense pas que Python y soit installé).
\item
  Sur votre ordinateur soyez sûr d'avoir les packages/modules déjà
  installés (dommage de perdre du temps pour les télécharger lors de
  l'exam).
\item
  Vous pouvez utiliser toutes les ressources à votre disposition !
\item
  Qu'est ce qu'il y a dans l'examen ?

  \begin{itemize}
  \tightlist
  \item
    Chercher les fautes dans un script R et Python.
  \item
    Quelques petits exercices faisant intervenir des boucles et des
    conditions.
  \item
    Création/modification de variables dans une base de données
    (character et numeric).
  \item
    Répondre et me rendre un .html généré via Rmarkdown de manière à ce
    que je puisse voir le code et l'output directement.
  \end{itemize}
\end{itemize}

\hypertarget{projet}{%
\section{Projet}\label{projet}}

\begin{itemize}
\item
  \textbf{Rendu le 22 Mars à minuit dernier délais /!\symbol{92} sinon 0 | Coeff : 1/3 (2ECTS)}
\item
  Comme précisé en cours vous devez créer un outil qui automatisera une
  ou plusieurs tâches, faites-vous plaisir.
\item
  Regroupez vos fonctions dans des scripts dédiés à leurs utilités (par
  exemple si vous avez un ensemble de fonctions dédiées au cleaning de
  data, mettez toutes ces fonctions dans un script clean\_data.py/R et
  importez-les depuis le script principal de votre outil).
\item
  Le projet doit être rendu disponible sur votre GitHub le 22 Mars à
  minuit.
\item
  Vous devez rediger un README détaillé pour expliquer le but de votre
  outil et comment le faire fonctionner. Je dois pouvoir comprendre
  votre projet et sa structure à partir du readme.
\end{itemize}

\hypertarget{oral-20-min}{%
\section{Oral (\textasciitilde{} 20 min)}\label{oral-20-min}}

\begin{itemize}
\item
  \textbf{23/24 Mars 9h-12h | Coeff : 1/3 (2ECTS)}
\item
  Venez avec votre ordinateur prêt à nous faire une démonstration de
  votre outil.
\item
  Expliquez clairement votre projet aux autres étudiant.es,
  permettez-leur de ressortir de ces oraux avec de nouvelles idées de
  projet.
\item
  Créez des slides avec Rmarkdown dans lesquelles vous parlerez de la
  structure de votre répertoire sur GitHub en explicitant les
  differentes étapes de l'élaboration de votre outil.
\item
  Montrez les parties du code qui vous semblent importantes et/ou qui
  vous ont posé problème, donnez-nous des armes pour ne pas avoir les
  mêmes difficultés que vous, si l'envie nous prend un jour de faire un
  projet similaire.
\end{itemize}

\hypertarget{but-du-cours-ce-que-vous-devez-retenir}{%
\section{But du cours / Ce que vous devez
retenir}\label{but-du-cours-ce-que-vous-devez-retenir}}

\begin{itemize}
\item
  Vous exposer à differents outils pour vous montrer qu'ils existent et
  vous initier à leur utilisation.
\item
  Savoir coder c'est surtout savoir utiliser google et stack overflow.
\item
  Vous pouvez faire tout ce que vous voulez avec la programmation et
  quelqu'un la certainement déjà fait avant vous.
\item
  Il faudra à l'issue du cours être en mesure de :

  \begin{itemize}
  \tightlist
  \item
    Créer des processus itératifs (boucles) et mettre en place des
    conditions.
  \item
    Utiliser/modifier les differents types d'objects dans R et Python
    (list, dict, matrix, tibble, dataframe ..)
  \item
    Rendre votre code plus flexible en utilisant des fonctions. (Exo
    chap 1/2/3 pour les exemples)
  \item
    Manipuler grossièrement des chaines de caractères et comprendre que
    certains schemas peuvent être repérés pour extraire/nettoyer
    facilement de l'information. (Exo chap3/4 pour les exemples)
  \item
    Comprendre que les mêmes tâches peuvent être réalisées de
    différentes manières et que certaines méthodes sont plus efficaces
    que d'autres.
  \item
    Savoir rendre disponible vos codes sur GitHub et savoir communiquer
    leurs outputs.
  \end{itemize}
\end{itemize}

\hypertarget{si-vous-navez-pas-tout-compris}{%
\section{Si vous n'avez pas tout
compris}\label{si-vous-navez-pas-tout-compris}}

\begin{itemize}
\item
  Vous ne devez pas connaitre mon cours sur le bout des doigts, cela n'a
  aucun sens puisqu'internet vous sera accessible le jour de l'examen !
\item
  Je reste disponible pour vous accompagner durant le mois qui séparera
  le derniers cours et l'examen, contactez-moi ! (Vraiment!)
\item
  Ceci est votre premier cours de programmation, pas d'inquiétude vous
  allez progresser et être à l'aise à l'issue de votre master. Gardez
  cette image en tête :
\end{itemize}

\centering

\includegraphics[width=0.6\textwidth,height=\textheight]{ilovecoding.JPG}

\end{document}
